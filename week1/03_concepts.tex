\begin{frame}
\frametitle{Lenguajes de programaci\'on}
    \begin{itemize}
        \item Un lenguaje formal  \pause
        \item Dise\~nado para comunicar instrucciones a la m\'aquina \pause
        \item Para expresar algoritmos. \pause
    \end{itemize}
    \emph{ Un leguaje regular!}
\end{frame}


\begin{frame}
    \frametitle{Algoritmos}
    Un procedimiento:
    \begin{itemize}
        \item Las acciones a ejecutar
        \item El \'orden
    \end{itemize}
\end{frame}

\note{
    La ducha. El orden es importante.
}

\begin{frame}
    \frametitle{El lenguaje de programaci\'on Python}
    Dos grupos grandes:
    \begin{itemize}
        \item Compilados: C, C++, C\#, $\ldots$
        \item Interpretados: Ruby, Bash, Perl, Python, $\ldots$
    \end{itemize}
    Python es un lenguaje \emph{orientado a objetos} interpretado.
\end{frame}


\note{
    What is the difference between these two languages:
    \begin{itemize}
        \item Does not need to be built
        \item Interpretetd is much much slower
        \item Compilation defines the machine.
    \end{itemize}
    Explain object orientation
}

\begin{frame}
    \frametitle{Python es}
    \begin{enumerate}
        \item Simple
        \item Es ``Free'' as in freedom, not ``free'' beer. \pause
        \item Es de alto nivel \pause
        \item Portablo \pause
        \item Interpretado \pause
        \item Orientado a objetos \pause
        \item Extensible \pause
    \end{enumerate}
\end{frame}

\subsection{Instalando Python}

\begin{frame}
    \begin{enumerate}
        \item Vamos a usar Python  en su versi\'on 3
        \item Vamos a usar jupyter notebook: Usamos el navegador como editor
    \end{enumerate}
\end{frame}

\begin{frame}
    \frametitle{Instalando python}
    \begin{itemize}
        \item Instalar Anaconda (una distribuci\'on de python. Incluye muchas cosas) 
        \item \url{https://www.continuum.io/downloads} (Bajar la version 3.5)
    \end{itemize}
    Instalar jupyter notebook
    \begin{itemize}
        \item Abrir una terminal (En Mac: Utilities->Applications, en Windows: Tecla Win + R, en GNU/Linux)
        \item coda install jupyter
    \end{itemize}
\end{frame}


\begin{frame}
    \frametitle{Probemos nuestro editor y nuestro lenguaje}
    \begin{itemize}
        \item Abrir una terminal (En Mac: Utilities->Applications, en Windows: Tecla Win + R, en GNU/Linux)
        \item jupyter notebook
    \end{itemize}
\end{frame}


\begin{frame}[fragile]
\frametitle{El primer programa}
\begin{python}
    print 'Hello World!'
\end{python}
Para ejecutar, apretar \texttt{Shift} y \texttt{Enter}.
\end{frame}


\begin{frame}
    \frametitle{Modos de operaci\'on de python}
    \begin{itemize}
        \item Interactivo \pause
        \item Script.
    \end{itemize}
\end{frame}

\begin{frame}
    \frametitle{Interactivo}
    \begin{itemize}
        \item Cada sentencia o conjunto de setencias es independiente
            \begin{itemize}
                \item En la consola: ejecutar \texttt{python}
                \item En el notebook
            \end{itemize}
    \end{itemize}
\end{frame}

\begin{frame}
    \frametitle{Script}
    \begin{itemize}
        \item Todas las sentencias se eval\'uan en conjunto
            \begin{itemize}
                \item Crear el programa en un archivo
                \item Cambiar la extensi\'on a \texttt{.py}
                \item Ejecutar el script: \texttt{python <archivo>.py}
            \end{itemize}
    \end{itemize}
\end{frame}

\subsection{Variables, expresiones y tipos}

\begin{frame}
    \begin{enumerate}
        \item Valor y tipo \pause
        \item Variables \pause
        \item Operaciones \pause
    \end{enumerate}
\end{frame}


\note{
    \begin{itemize}
        \item A esto hacemos detalle. explicamos el concepto de tipo de dato. 
        \item Enlazar con al representacion binaria de los datos 
        \item Mencionar las clases, mencionar que todos los tipos son clases y que por lo tanto son extensibles. El concepto de clase mencionamos rapidamente.
        \item Tenemos que mencionar que pasa con las variables y las operaciones. Sumar cadenas? Sumar enteros? dividir enteros? Representacion de numeros.
        \item Mencionar que no todos los elementos son identificaroes de variables validos.
        \item Explicar que hay tipos de variables que son fijos en otros lenguajes.
    \end{itemize}
}

\begin{frame}[fragile]
    Dado:
    \begin{python}
        a = 2
        b = 2.0
        c = '.'
    \end{python}
    Que resulta de:
    \begin{python}
        a/3
        b/3
        b + c
    \end{python}
\end{frame}

\begin{frame}[fragile]
    \frametitle{Cadenas}
    \begin{itemize}
        \item Una cadena es una \emph{sequencia} de car\'acteres.
        \item Uno puede referirse a un car\'acter en particular, empezando a 0 hasta n - 1
        \item La longitud de la cadena se mide con \texttt{len}
            \begin{python}
                a = 'horacio'
                print a[0]
                print len(a)
            \end{python}
    \end{itemize}
    \begin{python}
        print a[1.3]
        print a[10]
    \end{python}
\end{frame}


\begin{frame}[fragile]
    \frametitle{Condiciones}
    Si quiero indicar condiciones:
    \begin{itemize}
        \item La palabra clave \textbf{if} seguida de la condici\'on:
            \begin{python}
                if temperatura == 45:
                    print 'Hace mucho calor'
            \end{python}
    \end{itemize}
\end{frame}

\begin{frame}
    \frametitle{Ejercicios}
    \begin{enumerate}
        \item Resultado de \texttt{2 / 3} es 0. ?`Por qu\'e?
        \item Calcular la distancia que recorre una persona que corre a 12 km/h en 2.4 horas.
        \item Tigo me ofrece 100Gb. de datos a 10Mbps. Si bajo constantemente datos a la maxima velocidad, 
            ?`En cuanto tiempo llego al l\'imite? (10 Mbps son 10.000.000 bits segundo. 100 Gb son 100.000.000.000.000 bits)
        \item El vol\'umen de una esfera esta dado por $\frac{4}{3}\pi r^3$ donde r es el radio. Calcular.
    \end{enumerate}
\end{frame}

\begin{frame}
    \vspace{15mm}
    Lista de ejercicios: Ej_introduccion.pdf (\emph{en el classroom})
\end{frame}


\subsection{Estructuras de control}


\subsection{Estructuras de datos}


\begin{frame}
    \frametitle{Listas}
\end{frame}


\begin{frame}[fragile]
    \frametitle{Funciones}
    Una funci\'on es un conjunto de setencias que realiza una tarea espec\'ifica.
    \begin{python}
        def calcular_volumen(radio):
            return (4/3)*3.14*(r**3)
    \end{python}
\end{frame}

\begin{frame}
    \frametitle{Ciclos}
\end{frame}

\note{
    Definir el concepto de bloque en python y como los espacios determinan esto.\\
    Explicar las partes de la funcion.

}

\begin{frame}
    \frametitle{Ejercicios}
    \begin{enumerate}
        \item Definir el procedimiento histograma() que produce in histograma. Por ejemplo, histograma(1) es *, histograma(3) es ***
        \item Definir una funci\'on reverso(cadena) que invierte una cadena. Por ejemplo reverso('horacio') es 'oicaroh'
        \item Definir una funci\'on capicua(nombre) que retorna 'Si' si nombre es capicua. Una palabra capicua se lee de ambos lados igual, por ejemplo Anana
        \item Escribir una funci\'on que compute la longitud de una cadena sin usar la funci\'on len.
        \item Definir una funci\'on que suma los elementos de una lista.
        \item Definir una funci\'on que multiplica los elementos de una lista.
    \end{enumerate}
\end{frame}

\begin{frame}
    \frametitle{Tarea para la casa:}
    Leer el cap\'itulo 1.
\end{frame}
