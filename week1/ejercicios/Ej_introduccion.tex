
\documentclass[12pt]{article}
\usepackage{wasysym}
\usepackage[utf8]{inputenc}
\usepackage[T1]{fontenc}


\title{Lista de ejercicios: Ej\_introduccion.pdf}

\author{
    Dr. Horacio Caniza Vierci.\\ 
    Adaptado de \emph{C - How to program. Deitel \& Deitel. 4th edition}\\
        \underline{2016}
}
\date{}

\begin{document}

\maketitle

\begin{enumerate}
    \item Escribir los siguientes programas:
    \begin{itemize}
           \item Imprimir el mensaje: \texttt{Ingresar dos n\'umeros}
           \item Asignar el producto de dos variables \texttt{b} y \texttt{c} a una variable \texttt{a}
           \item Leer tres valores del teclado y guardarlos en variables.
    \end{itemize}
    \item Escribir un programa que reciba dos n\'umeros del usuario y calcule: la suma, el producto, el cociente y el resto de la divisi\'on.
    \item Escribir un programa que obtenga del usuario dos enteros e imprima el mayor seguido de \emph{es mas grande}. Si los n\'umeros son 
        iguales, que imprima \emph{son iguales}.
    \item Escribir un programa que lea el radio de un circulo e imprima su circumferencia y \'area. Definir $\pi$ como constante.
    \item Escribir un programa que imprima las siguientes formas:
        \begin{verbatim}
            *****     ***       *     *
            *   *   *     *    ***   * *
            *   *   *     *   ***** *   *
            *   *   *     *     *    * *  
            *****     ***       *     *
        \end{verbatim}
    \item Escribir un programa que lea 5 enteros e imprima el mayor y el menor.
    \item Escribir un programa que lea un entero y determine si es par o impar. 
    \item Escribir tus iniciales en letras de bloque. Construir cada letra de bloque con la letra 
        respectiva. Ejemplo:
        \begin{verbatim}
        HHHHHHH
           H
           H
        HHHHHHH

         CCCCCC
        C      C
        C      C

             V
           V
         V
           V
             V

        \end{verbatim}

    \item Escribir un programa que lea dos enteros y determine si el primero es \'ultiplo del segundo
    \item Escribir el siguiente patr\'on:
        \begin{verbatim}
        * * * * * * * 
         * * * * * * 
        * * * * * * *
         * * * * * * 
        * * * * * * *
         * * * * * * 
        * * * * * * *
         * * * * * * 
        \end{verbatim}
    \item Separar los d\'igitos en el n\'umero \texttt{475522}.
    \item Escribir un programa que produzca una tabla de cubos y cuadrados para todos los n\'umeros del 1 al 10.
\end{enumerate}

Desaf\'ios:

\begin{enumerate}
    \item Crear una aplicaci\'on que reciba la altura y el peso y calcule el BMI (Body Mass Index). El programa debe
        a su vez imprimir la siguiente tabla:
        \begin{verbatim}
        Valores BMI: 
        Bajo peso: < 18.5
        Normal: >= 18.5,  <= 24.9
        Sobrepeso: >= 25, <= 29.9
        Obeso: >= 30 
        \end{verbatim}
        La f\'ormula para el BMI es: $$ BMI = \frac{pesoKg}{alturaMetros^2} $$ 
    \item Comparir auto ahorra costos. Crear una aplicaci\'on que calcule tu costo diario de uso del auto y determine el 
        ahorro cuando se comparte el coche entre 2, 3 y 4 personas. El costo de operaci\'on del coche no esta relacionado al
        n\'umero de gente que lo usa.
        La aplicaci\'on debe considerar la siguiente informaci\'on:
        \begin{itemize}
            \item Kil\'ometros por d\'ia.
            \item Costo por litro de combustible
            \item Litros de combustible consumidos por cada km conducido.
            \item Costos de estacionamiento
        \end{itemize}
\end{enumerate}

\end{document}
