

\documentclass[12pt]{article}
\usepackage{wasysym}
\usepackage[utf8]{inputenc}
\usepackage[T1]{fontenc}

% Default fixed font does not support bold face
\DeclareFixedFont{\ttb}{T1}{txtt}{bx}{n}{12} % for bold
\DeclareFixedFont{\ttm}{T1}{txtt}{m}{n}{12}  % for normal

% Custom colors
\usepackage{color}
\definecolor{deepblue}{rgb}{0,0,0.5}
\definecolor{deepred}{rgb}{0.6,0,0}
\definecolor{deepgreen}{rgb}{0,0.5,0}

\usepackage{listings}

% Python style for highlighting
\newcommand\pythonstyle{\lstset{
        language=Python,
        basicstyle=\ttm,
        otherkeywords={self},             % Add keywords here
        keywordstyle=\ttb\color{deepblue},
        emph={MyClass,__init__},          % Custom highlighting
        emphstyle=\ttb\color{deepred},    % Custom highlighting style
        stringstyle=\color{deepgreen},
        frame=tb,                         % Any extra options here
        showstringspaces=false            % 
}}


% Python environment
\lstnewenvironment{python}[1][]
{
    \pythonstyle
    \lstset{#1}
}
{}

% Python for external files
\newcommand\pythonexternal[2][]{{
        \pythonstyle
\lstinputlisting[#1]{#2}}}

% Python for inline
\newcommand\pythoninline[1]{{\pythonstyle\lstinline!#1!}}



\title{Lista de ejercicios: Ej\_control\_de\_flujo.pdf}

\author{
    Dr. Horacio Caniza Vierci.\\ 
    Adaptado de \emph{C - How to program. Deitel \& Deitel. 4th edition}\\
        \underline{2016}
}
\date{}

\begin{document}

\maketitle

    \begin{enumerate}
        \item Escribir un programa qeu encuentre el m\'inimo de varios enteros. El primer par\'ametro determina la cantidad
            de n\'umeros que ser\'an leidos.
        \item Escribir un programa que calcule la suma de los enteros pares desde 2 hasta 30.
        \item Escribir un programa que calcule el producto de los enteros impares desde 1 hasta 30.

        \item Escribir un programa que imprima las siguientes formas:
            \begin{verbatim}

            (a)
            *
            **
            ***
            ****
            *****
            ******
            *******
            ********

            (b)
            ********
            *******
            ******
            *****
            ****
            ***
            **
            *

            (c)
            ********
             *******
              ******
               *****
                ****
                 ***
                  **
                   *
            \end{verbatim}

        \item Escribir un programa que imprima histogramas. El programa debe recibir cinco n\'umeros (min. 1, m\'ax 30) y por cada n\'umero
            imprimir una secuencia de tantos * c\'omo valor tenga el n\'umero.

        \item Escribir un programa que determine el valor de $\pi$ usando el hecho  $ \pi = 4 - \frac{4}{3} + \frac{4}{5} - \frac{4}{7} + \frac{4}{9} -\frac{4}{11} + \ldots$. Indicar el n\'umero de t\'erminos necesarios para obtener: 
            \begin{itemize}
                \item 3.14
                \item 3.141
                \item 3.1415
                \item 3.14159
            \end{itemize}

        \item Una tripleta de Pit\'agoras esta compuesta de tres enteros que
            determinan los lados de un triangulo. Los tres lados debn
            satisfacer el criterio de que la suma de los cuadrados de dos de
            los lados debe ser igual al cuadrado del tercero (el teorema de
            Pit\'agoras). Usar tres loops anidados para encontrar todas las triplas.
            ?`Es posible encontrar todas las triplas de esta manera?  ?`Cu\'anto tiempo 
            es necesario para encontrar todos las triplas de esta manera?
        \item 
            Las leyes de De Morgan se expresan en castellano:
            \begin{itemize}
                \item La negaci\'on de una conjunci\'on es la disjunci\'on de las negaciones.
                \item La negaci\'on de una disjunci\'on es la conjunci\'no de las negaciones. 
            \end{itemize}
            Usar las leyes de De Morgan para encontrar representaciones alternativas para las siguientes expresiones:
            \begin{python}
                not( x < 5 ) and not (y > 7)
                not (a == b) or not (g != 5)
                not ( (x <= 8) and (y > 4) )
                not ( (i > 4 ) or ( j <= 6) )
            \end{python}

        \item Escribir un programa que imprima diamantes de tama\~no arbitrario. 
           Es suficiente con construir diamantes impares.A modo de ejemplo, el siguiente es un diamante de tama\~no 9:
            \begin{verbatim}
                    *
                   ***
                  *****
                 *******
                *********
                 *******
                  *****
                   ***
                    *
            \end{verbatim}

        \item Escribir un programa que imprima una tabla de n\'umeros enteros del 1 al 100 y su respectivo
            n\'umero romano.

    \item Los conductores estan preocupados por el consumo por kil\'ometro de sus
        veh\'iculos. Escribur un programa que reciba como entrada el total de
        kil\'ometros conducido y el total de combustible necesario para llenar el
        tanque de sus veh\'iculos. El programa tiene que calcular en indicar al
        ausuario los il\'ometros por litro obtenidos por cada tanque lleno. Al terminar
        el programa, indicar al usuario el promedio total de consumo en litros y el
        rendimiento promedio en litros por kil\'ometro.
        A modo de ejemplo:
        \begin{verbatim}
        El combustible cargado (-1 para terminar):  10
        Km conducidos: 100 km.
        Km/litro: 10

        El combustible cargado (-1 para terminar):  12
        Km conducidos: 100 km.
        Km/litro: 8.3


        El combustible cargado (-1 para terminar):  -1
        Km conducidos: 200 km
        Km/litro: 9.15

        \end{verbatim}

    \item Escribir un programa que determine si un cliente ha excedido su l\'imite de cr\'edito. Para cada cliente, 
        los siguientes datos son necesarios:
        \begin{itemize}
            \item N\'umero de cuenta
            \item Balance al 1 del mes
            \item Total pagado
            \item Total cr\'edito solicitado por el cliente
            \item L\'imite de cr\'edito aprobado
        \end{itemize}

        El programa debe calcular el nuevo balance (inicial + total pagado -  cr\'edito solicitado)
        Ej:
        \begin{verbatim}
        Cuenta (-1 para terminar): 100
        Balance al 1 del mes: 5394
        Total pagado: 1000
        Credito solicitado: 500
        Limite de credito: 5500
        ****
        LIMITE EXCEDIDO:
        Cuenta: 100
        Limite de credito: 5500
        Balance: 5894
        ****


        Cuenta (-1 para terminar): 233
        Balance al 1 del mes: 1000
        Total pagado: 123
        Credito solicitado: 321
        Limite de credito: 1500
        

        Cuenta (-1 para terminar): -1
        \end{verbatim}

    \item Una f\'abrica que produce productos qu\'imicos para la industria pl\'astica paga a sus empleados de venta
        por comisi\'on. La persona recibe 1.000.000 por mes, m\'as 9\% de comisi\'on sobre el total de sus ventas. Ej.
        Si el vendedor A vende por valor de 5.000.000, su salario es: 1.000.000 m\'as 450.000. Escribir un programa que 
        recibiendo las ventas por cada vendededor, calcule su salario final.

    \item El inter\'es en un prestamo se calcula como:  $int = monto \times tasa \times \frac{dias}{365}$. Escribir un programa
        que  dado un monto solicitado, calcule el costo total del prestamo al final del ciclo.

    \item Desarrollar un programa que caclule el pago a empleados. La empresa paga tiempo total a las primeras 40 horas trabajadas
        por el empleado y tiempo y medio por cada hora sobre las primeras 40. El programa debe recibir el n\'umero total de horas 
        trabajas y la tarifa horaria por cada empleado.

    \item Escribir un programa que imprima n\'umeros del 1 al 10 con 3 espacios en medio de cada uno.

    \item Escribr un programa que encuentre el m\'aximo en una serie de n\'umeros. El programa debe leer
        del usuario los n\'umeros y luego indicar el mayor. 

    \item Escribir un programa que imprima la siguiente tabla:
        \begin{verbatim}
        N   10*N    100*N   1000*N
        1   10      100     1000
        2   20      200     2000
        3   30      300     3000
        4   40      400     4000
        5   50      500     5000
        6   60      600     6000
        7   70      700     7000
        8   80      800     8000
        9   90      900     9000
        10  100     1000    10000
        \end{verbatim}

    \item Esribir un programa que use un loop para imprimir:
        \begin{verbatim}
        A   A+2 A+3 A+4
        3   5   6   7
        6   8   9   10
        9   11  12  13
        12  14  15  16
        15  17  18  19
        \end{verbatim}

    \item Escribir un programa que verifique si un n\'umero es capicua.
    \item Imprimir el n\'umero decimal codificado por un n\'umero binario.
    \item Podemos determinar la velocidad de nuestra computadora. Escribir un programa
        que cuente de 1 a 300,000,000 (trescientos millones) de 1 a 1. Cada vez que se llegue a m\'ultiplos de 100,000,000
        imprimir el n\'umero en pantalla. Determinar el tiempo que toman 100,000,000 de iteraciones con el reloj. Calcular
        cuando tomar\'ia 1,000,000,000,000 iteraciones.

    \item Imprimir m\'ultiplos de 2 en un ciclo infinito. Antes de correr el programa, ?`que va a pasar?
    \item Escribir un programa qeu imprima el di\'ametro, circunferencia y \'area de un circulo de radio R. 
    \item Escribir un programa que determine si 3 valores dados pueden ser usados para representar un tri\'angulo. Ej. 2, 1 y 4 no pueden pero
        1, 1 y 1 si pueden.
    \item Escribir un programa que determine si 3 valores dados pueden ser usados para representar un tri\'angulo rect\'angulo. Ej. 2, 1 y 4 no pueden pero
        1, 1 y 1 si pueden.
    \item El factorial de un n\'umero, escrito $n!$ se define como: $n! = n.(n-1).(n-2).\ldots.1$ y $0! = 1$. El factorial no est\'a definido para n\'umeros negativos.
        \begin{itemize}
            \item Escribir un programa que calcule el factorial de un n\'umero.
            \item Escribir un programa que estime la consntante $e$ (el n\'umero de euler) con la f\'ormula: $e = 1 + \frac{1}{1!} + \frac{1}{2!} + \ldots$
            \item Escribir un programa que calcule $e^x$ por medio de $ e^x = 1 + \frac{x}{1!} + \frac{x^2}{2!} + \frac{x^3}{3!} + \ldots $
        \end{itemize}
        ?`Qu\'e dificultad puede surgir al calcular el factorial?
\end{enumerate}

Desaf\'ios:
\begin{enumerate}
    \item Buscar en internet la poblaci\'on actual mundial y su tasa de crecimiento. Escribir un programa que determine la cantidad de gente
        en funci\'on a estos par\'ametros en dos, tres, cinco y diez a\~nos.
    \item Existen monitores cardiacos que pueden ser utilizados mientras uno se ejercita. De acuerdo a la AHA (Americal Heart Association) la f\'ormula
        para determinar el ritmo m\'aximo es 220 menos tu edad en a\~nos (Consulte a su m\'edico antes de modificar su plan de ejercicio o empezar a hacer ejercicio). La AHA estima que el ritmo ideal es 50-85\% del ritmo m\'aximo. Crear un programa que lea el d\'ia, mes y a\~no de nacimiento de la persona y caclque la edad de la persona, el ritmo m\'aximo y el ritmo ideal.
    \item Escribir un programa que encripte informaci\'on. La aplicaci\'on va a recibir un n\'umero de 4 d\'igitos. Cada d\'idigito debe ser reemplazado con el resto de la suma del d\'igito original m\'as 7 dividido por 10. Imprimir el n\'umero encriptado. Escribir un programa separado que reciba un n\'umero encriptado y lo desencripte.  A modo de ejemplo, el usuario ingresa 3. Este n\'umero debe ser reemplazado por 0, porque: 3+7 = 10, 10\%10 = 0.
\end{enumerate}
\end{document}
