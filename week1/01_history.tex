
%1 hr.
\begin{frame}
    \frametitle{?`Qu\'e es la computaci\'on?}
    Computaci\'on es:
    \begin{itemize}
        \item Una tarea con objetivo definido que requiera, se beneficie de o cree computadoras \cite{ACM1}
        \item Entonces... \pause
            \begin{itemize}
                \item Construir hardware y software \pause
                \item Sistemas inteligentes, Comunicaciones, Juegos.  
                \item $\ldots$
            \end{itemize}
    \end{itemize}
    \pause
    ?`Y la computadora? ?`Qu\'e es la computadora?
    \begin{itemize}
        \item Una computadora no computa (!`no hace aritm\'etica!) \pause
        \item Una computadora es un \emph{data handler}!
    \end{itemize}
\end{frame}

\begin{frame}{Historia}
    Babbage and the lot.
    Insert timeline of computation here and discuss each step
\end{frame}

\begin{frame}{Dise\~nemos una computadora humana}
    Tenemos 4 operarios de archivo: \pause
    \begin{enumerate}
        \item Lento, pero conoce todas las operaciones. \pause
        \item 10 veces m\'as r\'apido, pero solo sabe sumar. \pause
        \item 10 veces m\'as r\'apido, pero no sabe sumar. \pause
        \item 10 veces m\'as r\'apido, pero no sabe siquiera leer.\pause
    \end{enumerate}
    \textcolor{red}{Una computadora general!} 
\end{frame}

\begin{frame}
    \frametitle{Representaci\'on binaria, \emph{aka} Computers are stupid}
    \begin{itemize}
       \item Armemos un sistema con colores: Azul y Rojo.
       \item !`Ejercicio! 
           \begin{itemize}
               \item Dise\~nar un sistema para identificar cuartos en un hotel usando dos colores. Establecer los l\'imites.
           \end{itemize}
    \end{itemize}
\end{frame}

%1.5
\begin{frame}{Arquitectura de sistemas modernos}
    \begin{itemize}
        \item ''First Draft of a Report on the EDVAC`` primavera de 1945. \pause 
            (\emph{aka} Steve Jobs vend\'ia un producto de m\'as de 60 a\~nos.) \pause
        \item Cuatro partes:
            \begin{enumerate}
                \item La Unidad Aritm\'etico-L\'ogica (ALU): realiza las operaciones.\pause
                \item La Unidad de Control Central(CU): controla y sincroniza \pause
                \item La memoria (M): almacena datos e instrucciones \pause
                \item Dispositivos de Input/Output (IO)
            \end{enumerate}
    \end{itemize}
\end{frame}

% SO 1

\begin{frame}{La computadora moderna}
    \begin{itemize}
        \item Uno o m\'as procesadores
        \item Sonido (entrada y salida), video (posiblemente varios), conexi\'on a redes, autenticaci\'on $\ldots$
    \end{itemize}
    \pause
    Una capa de \emph{software} que presenta un modelo simplificado del \emph{hardware}:
    \begin{itemize} 
        \item Windows, Mac, Android, IOS, Blackberry OS. \pause
        \item GNU/Linux, GNU Hurd, Minix, Solaris, LynxOS, Hp UX, $\ldots$ \pause
        \item ?`Y las m\'aquinas en una f\'abrica?
    \end{itemize}
    \pause
    \textcolor{red}{Sistema Operativo}
\end{frame}


\begin{frame}
    \frametitle{El sistema operativo como m\'aquina extendida}
    \begin{itemize}
        \item?`C\'omo imprimimos en la pantalla? \pause
        \item La abstracci\'on es esencial para manejar la complejidad. \pause
        \item Varios niveles de abstracci\'on
    \end{itemize}
\end{frame}

\begin{frame}
    \frametitle{Bajo nivel: C\'odigo m\'aquina} 
        b8    21 0a 00 00   \#moving "!n" into eax \\
        a3    0c 10 00 06   \#moving eax into first memory location \\
        b8    6f 72 6c 64   \#moving "orld" into eax \\
        a3    08 10 00 06   \#moving eax into next memory location \\
        b8    6f 2c 20 57   \#moving "o, W" into eax
        a3    04 10 00 06   \#moving eax into next memory location \\
        b8    48 65 6c 6c   \#moving "Hell" into eax \\
        a3    00 10 00 06   \#moving eax into next memory location \\
        b9    00 10 00 06   \#moving pointer to start of memory location into ecx \\
        ba    10 00 00 00   \#moving string size into edx \\
        bb    01 00 00 00   \#moving "stdout" number to ebx \\
        b8    04 00 00 00   \#moving "print out" syscall number to eax \\
        cd    80            \#calling the linux kernel to execute our print to stdout \\
        b8    01 00 00 00   \#moving "sys\_exit" call number to eax \\
        cd    80            \#executing it via linux sys\_call \\
        \pause \par
        \textcolor{red}{Resultado} \pause Imprime 'Hello World' en pantalla.
\end{frame}

\begin{frame}
\frametitle{Bajo/Medio nivel: Ensamblador}
    org  0x100       \\
    mov  dx, msg      \\
    mov  ah, 9        \\
    int  0x21         \\
    mov  ah, 0x4c     \\
    int  0x21         \\
    msg  db 'Hello World', 0x0d, 0x0a, '$'   ; $-terminated message \\
    \pause \par
    \textcolor{red}{Resultado} \pause Imprime 'Hello World' en pantalla.
\end{frame}

\begin{frame}
\frametitle{Alto nivel: Python}
    print('Hello World') \\
\pause
\textcolor{red}{Resultado} \pause Imprime 'Hello World' en pantalla.
\end{frame}

\begin{frame}
    \frametitle{El sistema operativo como administrador de recursos}
    \begin{block}{Escenario}
        Estoy jugando Clash of Clans y suena el FaceTime. ?`Qu\'e pasa?
    \end{block}
    \pause
    Multiplexar (Del Latin \emph{Multi} - Muchos, \emph{plexos} - Lados, dobleces)
    \begin{itemize}
        \item Tiempo \pause
        \item Espacio
    \end{itemize}
\end{frame}

\begin{frame}
    Mi computadora:
    \begin{itemize}
        \item GNU/Linux Debian 8: \pause 420.000.000 millones de instrucciones.
        \item El costo estimado: \pause 19.000 Millones de US\$.  
    \end{itemize}
    ?`Y en perspectiva?
        url{http://www.informationisbeautiful.net/visualizations/million-lines-of-code/}
\end{frame}

